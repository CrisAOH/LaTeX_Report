\section{Competencias interpersonales}
Las competencias interpersonales son aquellas que permiten expresar los propios sentimientos. Tienden a facilitar los procesos de interacción social y cooperación.

Entre las competencias interpersonales aplicadas se encuentran:
 \begin{itemize}
    \item Capacidad crítica y de autocrítica: Tener la capacidad de analizar nuestros actos y habilidades, así como las de los demás, es importante para poder ver oportunidades de mejora y de éxito.
    \item Trabajo en equipo: Saber trabajar con otras personas permite establecer mejores relaciones con los compañeros y un desarrollo más óptimo del proyecto.
    \item Compromiso ético: La ética es fundamental en nustras vidas, y debe aplicarse en cualquier aspecto. 
 \end{itemize}

Entre las competencias interpersonales desarrolladas se encuentran:
 \begin{itemize}
    \item Habilidades interpersonales: Considero que estas habilidades nunca dejan de desarrollarse, pues siempre tendremos que interactuar con diferentes personas y en diferentes entornos.
    \item Capacidad de comunicarse con profesionales de otras áreas: Una empresa siempre va a tener diferentes áreas de las cuales en algún momento se va a necesitar algo. Saber comunicarse con estas áreas también permite conseguir un desarrollo laboral mejor.
    \item Habilidad para trabajar en un ambiente laboral: Estos 5 meses de Residencias Profesionales me ayudaron a comenzar con mi integración al ambiente laboral. Este primer acercamiento no fue muy complicado, pero todavía falta experimentar otras situaciones.
 \end{itemize}