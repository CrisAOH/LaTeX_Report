\section{Competencias Sistemáticas}
Las competencias sistemáticas son las destrezas y habilidades que connciernen a los sistemas en su totalidad. Suponen una combinación de la comprensión, la 
sensibilidad y el conocimiento que permiten al individuo ver como las partes de un todo se relacionan y se estructuran y se agrupan. 

Entre las competencias sistemáticas aplicadas se encuentran: 
    \begin{itemize}
        \item Capacidad de aplicar los conocimentos en la práctica: La mayoría de conocimentos que aprendí durante mis clases de universidad los apliqué para el desarrollo de este proyecto. Además, también necesario saber aplicar los nuevos conocimientos que se iban adquiriendo durante este periodo.
        \item Habilidades de investigación: Saber investigar fue fundamental para encontrar la respuesta a las distintas incógintas que iban surgiendo durante el desarrollo.
        \item Capacidad de aprender: Para este proyecto tuve que aprender sobre nuevas tecnologías y herramientas. Sin esta capacidad, no hubiera podido desarrollar con éxito mis actividades.
        \item Capacidad de adaptarse a nuevas situaciones: A lo largo del proyecto ocurrían situaciones inseperadas, pero había la necesidad de adaptarse para comprenderals y saber como solucionarlas.
        \item Habilidad para trabajar de forma autónoma: No había compañeros que pudieran apoyarme con dudas qeu tuviera sobre los frameworks puesto que ellos no tenían el conocimiento lo que ocasionó que la mayoría del tiempo trabajar de manera autónoma.
        \item Preocupación por la calidad: Intentar conseguir una buena calidad para las actividades era fundamental para lograr la satisfacción del cliente y, a su vez, reducir la cantidad de errores que pudiera tener la aplicación.
        \item Búsqueda del logro: En las Residencias Profesionales también fue importante intentar sobresalir y alcanzar metas.
    \end{itemize}