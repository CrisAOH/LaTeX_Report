\chapter{Conclusiones y recomendaciones}
El periodo de 5 meses que realicé para mis Residencias Profesionales me permitió aprender muchas cosas nuevas y poner en práctica habilidades y conocimientos que ya tenía.

El ingresar a un entorno laboral me permitió empezar a conocer las diferentes maneras que hay de trabajar en una empresa de desarrollo de software, así como los procesos que se deben seguir para poder ingresar a una empresa.

Considero que, para mi, una de las cosas más difíciles de las Residencias Profesionales fue aplicar y desarrollar las habilidades blandas. El hecho de estar en un lugar nuevo con gente nueva es algo que, si bien no es nuevo, siempre es complicado porque no acostumbro a estar en situaciones así pero fue necesario superar estas dificultades para llevar una buena convivencia con los compañeros de trabajo y facilitar el trabajo en equipo.

Lo más importante que me llevo de esta experiencia fue todo el aprendizaje adquirido. Por un lado repasé tecnologías que llevaba mucho tiempo sin utilizar, como fue Express.js y los servicios en la nube de AWS, por otro lado aprendí nuevas tecnologías como fue Next.js y React. 

A pesar de que aprender Next.js y React son cosas positivas que me llevo de las Residencias Profesionales, en un principio sí fue otra de las cosas que se me complicó pues yo esperaba que para este proyecto utilizaría frameworks con los que ya estuviera familiarizado, como Laravel, pero una gran sorpresa me llevé cuando supe que utilizaría un framework completamente nuevo. Desafortunadamente no tuve con quien apoyarme para aprender y resolver dudas acerca de Next.js y React y tampoco tuve el tiempo suficiente para aprenderlo de una manera adecuada para un proyecto importante como Mezfer Insider pero considero que, de cierta manera, esto también fue bueno porque me permitió comprender que ser autodidacta no es complicado y lo puedo hacer.

En general, el periodo de Residencias Profesionales fue una experiencia bastante constructiva y positiva para mi. Me permitió comenzar a entender como funciona el mundo laboral y como debo prepararme para futuros empleos que pudiera tener, también me permitió conocer como es el trabajo de un Ingeniero en Sistemas Computacionales, la importancia de aplicar metodologías para el desarrollo de Software y mantenerse actualizado en conocimientos con las nuevas tecnologías. Otro aspecto importante que aprendí es que las habilidades blandas también son escenciales para un trabajo pues gracias a ellas se puede lograr una excelente convivencia con los compañeros y un gran trabajo en equipo.

Ahora, la principal recomendación que hago para la mejora del proyecto es que este actualice tanto el framework como las librerías que se usan a sus últimas versiones. Actualmente tanto el framework como las librerías han quedado muy desactualizadas y esto en algún momento podría causar problemas en cuanto a vulnerabilidades de la aplicación web; si en algún momento alguien llegara a detectarlas, podría aprovecharse de esto y realizar algún ataque que comprometa la información de la empresa y de los usuarios. Además, al actualizar el framework y las librerías se podría mejorar el rendimiento de la aplicación gracias a las diferentes mejoras que se hacen o también con nuevas características que llegan con cada actualización.

Otra recomendación que hago es establecer tiempos de entrega más realistas. En repetidas ocasiones llegó a pasar que a las actividades se les asignaba un tiempo de entrega muy corto y esto llegaba a afectar la calidad de lo que se entregaba. Aún cuando se realizaba la estimación de las actividades y se establecía que serían algo complicadas de realizar, se asignaban fechas de entrega muy cortas y esto comprometía la calidad del trabajo. 

