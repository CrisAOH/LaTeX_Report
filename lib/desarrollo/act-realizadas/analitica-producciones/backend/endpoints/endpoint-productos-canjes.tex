\subsubsection{Endpoint para obtener un top de los productos que más se canjean}
El query de este endpoint funciona de una manera bastante similiar al explicado en la sección anterior, la única diferencia esta en que ahora el top se aplica a productos y no usuarios.

Nuevamente participan las tablas de piezas de las producciones y los estatus de las piezas. También se une la tabla de productos para poder obtener correctamente el nombre del producto.

Con estas tres tablas es suficiente para obtener el nombre de los productos y el total de canjes que se han realizado, con este total de canjes se puede crear un top para conocer cual producto es el que mejor desempeño tiene en cuanto a canjes.