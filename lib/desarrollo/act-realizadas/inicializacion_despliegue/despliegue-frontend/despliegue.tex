\subsection{Despliegue del proyecto de Frontend en servidor de producción}
Para desplegar el proyecto en el servidor de producción se deben seguir casi los mismos pasos que con la inicialización, pero hay algunos pasos extra que deben realizarse.

Principalmente, deben de estar instalados Node.js y Yarn para poder instalar las dependencias y ejecutar el proyecto. El proceso de instalación es el mismo, no hay ninguna diferencia.

Otra herramienta importante que se debe instalar es PM2 (Process Manager 2), un gestor de procesos que permite mantener proyectos de Node.js en ejecución, es decir, si el servidor llegase a ser detenido o reiniciado, PM2 se encargará de iniciar el proyecto automáticamente sin la necesidad de que un administrador ingrese al servidor a iniciar todos los proyectos nuevamente. Para instalar PM2, se ejecuta el siguiente comando:
    \begin{center}
        \textbf{
            \emph{
                yarn global add pm2
                }
            }
    \end{center}
El siguiente paso es compilar el proyecto para que Next.js realice todas las optimizaciones necesarias y así quede preparado para producción. Para compilar el proyecto, desde la carpeta principal se ejecuta el siguiente comando:
    \begin{center}
        \textbf{
            \emph{
                yarn build
                }
            }
    \end{center}
De esta manera Next.js comenzará a compilar y optimizar todas las páginas que tenga la aplicación. Al finalizar el proceso, se creará una carpeta llamada ``.next'', la cual contiene el proyecto optimizado y listo para producción.

Ahora, hay 5 archivos importantes que deben ser enviados al servidor para poder ejecutar correctamente el proyecto:
    \begin{itemize}
        \item .next: Esta es la carpeta que contiene todo el proyecto optimizado.
        \item .env: Este es el archivo donde se almacenan las variables de entorno.
        \item package.json: Este es el archivo que contiene todas las dependencias que han sido instaladas en el proyecto.
        \item yarn.lock: Un archivo generado por Yarn que lleva un registro de las versiones exactas de las dependencias que requiere un proyecto para que funcione adecuadamente.
        \item next.config.js: Este es el archivo que contiene la configuración del servidor de Next.js.
    \end{itemize}
Para enviar estos archivos al servidor, se ejecuta el siguiente comando:
    \begin{center}
        \textbf{
            \emph{
                scp -i --/ruta/llave/PEM -r archivos usuario@ip\_{}publica\_{}servidor:--/carpeta/destino
                }
            }
    \end{center}
Una vez estén todos los archivos en el servidor de producción, se instalan las dependencias como se explicó anteriormente, y finalmente, se ejecuta el proyecto con PM2 para que este se encargue de gestionarlo. Para ejecutar el proyecto con PM2, se utiliza el siguiente comando:
    \begin{center}
        \textbf{
            \emph{
                pm2 start ``yarn start'' --name nombre-app
                }
            }
    \end{center}
Posteriormente, se guarda la configuración con el siguiente comando:
    \begin{center}
        \textbf{
            \emph{
                pm2 save
                }
            }
    \end{center}
Para hacer que la aplicación se ejecute automáticamente cuando se reinicie el servidor, se utiliza el siguiente comando:
    \begin{center}
        \textbf{
            \emph{
                pm2 startup
                }
            }
    \end{center}
Al ejectuar este comando PM2 proporcionará otro comando que solamente debe ser copiado, pegado y ejecutado, y se habrá terminado de configurar. Ahora, con cada reinicio del servidor, la aplicación se ejecutará automáticamente.

Cuando se quieran subir cambios al proyectos, simplemente se repiten los pasos:
    \begin{itemize}
        \item Compilar el proyecto.
        \item Subir los archivos que hayan sufrido cambios al servidor.
        \item Reiniciar el proceso de PM2.
    \end{itemize}
Para reiniciar el proceso de PM2, se ejecuta uno de los siguientes comandos:
    \begin{center}
        \textbf{
            \emph{
                pm2 reload nombre-app
                }
            }

        \textbf{
            \emph{
                pm2 restart nombre-app
            }
        }
    \end{center}
Y con esto quedarán aplicados los cambios que se hayan subido.