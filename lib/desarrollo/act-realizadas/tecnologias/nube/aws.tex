\subsection{Servicios en la nube}
La plataforma que se utilizó para proveer los servicios en la nube fue Amazon Web Services. De esta plataforma se utilizaron servicios como:
\begin{itemize}
    \item Amazon Elastic Compute Cloud (EC2): Es un servicio que proporciona capacidad de computación escalable bajo demanda en la nube. Permite lanzar cuantos servidores virtuales sean necesarios. Dos instancias de EC2 fueron utilizadas para lanzar el proyecto de Mezfer Insider y la API.
    \item Amazon Relational Database Service (RDS): Es un servicio web que facilita la configuración, la operación y la escala de una base de datos relacional en la nube. Se utilizó este servicio para crear la base de datos de Mezfer Insider.
    \item Amazon Simple Storage Service (S3): Es un servicio de almacenamiento de objetos que ofrece escalabilidad, disponibilidad de datos, seguridad y rendimiento líderes del sector. Este servicio fue utilizado para almacenar recursos importantes de Mezfer Insider, como archivos e imágenes.
\end{itemize}