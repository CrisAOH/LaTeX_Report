\subsection{Tecnologías de backend}
El backend consistió en desarrollar una API para que el frontend pudiera comunicarse con esta y realizar peticiones HTTP para interactuar con la base de datos.

La principal tecnologia de backend es Node.js, un entorno de ejecución utilizado para poder ejecutar código de JavaScript directamente en los servidores, es decir, sin la necesidad de un navegador web.

Junto a Node.js se utiliza Express.js, un framework de backend para Node.js que proporciona características y herramientas robustas para desarrollar aplicaciones de backend escalables. Sus herramientas como peticiones y respuestas HTTP, enrutamiento y middlewares resultaron muy útiles para el desarrollo de la API.