\subsection{Tecnologías de frontend}
Para facilitar el desarrollo del frontend, se decidió comprar una plantilla de pánel de usuario llamada DashTail, la cual proporciona vistas ya diseñadas o componentes para poder ir construyendo poco a poco la vista para el usuario; esta plantilla ofrece una alta personalización, por lo que es posible realizar cualquier cambio a los componentes.

DashTail fue desarrollada utilizando Next.js y Tailwind CSS. 

Next.js es un framework basado en JavaScript y React que permite crear aplicaciones web modernas, dinámicas y escalables. Aunque es un framework full-stack, en DashTail sólo se utiliza para frontend.

Tailwind CSS es un framework de CSS para el diseño de páginas web. Su principal característica es que no genera una serie de clases predefinidas para elementos, en su lugar, crea una lista de clases ``de utilidad'' que son utilizas para dar estilos individuales a cada elemento. Es gracias a este framework que los componentes de la plantilla son altamente personalizables.