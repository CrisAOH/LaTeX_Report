\subsubsection{Endpoint para actualizar las producciones}
En este endpoint pueden ocurrir dos cosas: actualizar información de la producción o solamente actualizar el estatus.

Primero se explicará la actualización de la información. Esto es muy similar a lo que ocurre cuando se crea una nueva producción. Se utiliza el mismo formulario, con la diferencia de que ahora los campos no estarán vacíos, sino que contendrán la información previa. Una vez que el usuario haya cambiado la información necesaria, el objeto será enviado al Backend para su procesaminto.

En el Backend, esta operación de actualización se realiza dentro de una transacción, dentro de esta se deben cumplir cierta condición para saber que acciones realizar. La condición es comprobar si la información de actualización contiene el campo que hace referencia al ID del estatus, lo que ocurre cuando si existe este campo en el objeto será explicado más adelante, pero por ahora, como no existe este campo, la información del objeto es utilizada para actualizar la información que ya se encontraba almacenada en la base de datos. Si ocurre algún error, todo el proceso se borrará.

Ahora, ¿qué ocurre cuando el objeto si tiene el campo de ID del estatus? Este objeto siempre vendrá solo en el objeto, pues la actualización de estatus ocurre cuando en el Frontend se presionan botones específicos para esto, cada uno con un estatus vinculado. Ahora que este campo existe, se debe validar que el ID sea el que corresponde al estatus de ``Aceptado''; si este corresponde, se realiza una búsqueda de la producción que se está modificando para poder obtener la cantidad de piezas que se van a producir. Para insertar todas las piezas se crea un arreglo vacío, y dentro de un ciclo que se detendrá hasta que se alcance el máximo de piezas, se ingresarán objetos con las información de las piezas (folio, producción, estatus, código QR), y cuando se termina el ciclo este arreglo se utiliza para insertar todas las piezas en la tabla correspondiente.

Cuando el ID del estatus no corresponde al de ``Aceptado'' no se realiza ninguna otra acción, simplemente se actualiza el estatus de la producción al que corresponda, esto también implica que ninguna pieza será creada. 