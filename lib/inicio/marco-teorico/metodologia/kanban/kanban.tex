\subsection{Metodología Kanban}
La metodología Kanban es un sistema de gestión de tareas que forma parte de las metodologías ágiles. Su propósito principal es supervisar y optimizar el flujo de trabajo desde el inicio hasta la finalización de las tareas, asegurando un proceso continuo y eficiente.

El primer paso para utilizar Kanban es crear un tablero visual que sea accesible para todo el equipo, donde se representen las diferentes fases del flujo de trabajo. El tablero debe tener columnas que indiquen el estado de las tareas, desde que inician hasta que terminan. Cada tarea es representada mediante tarjetas que se mueven a través del tablero a medida que avanza por las diferentes etapas del proceso.

La metodología Kanban es ideal para empresas que requieren flexibilidad y eficiencia en la gestión de sus proyectos. Al proporcionar una visión clara del flujo de trabajo y facilitar la priorización de tareas, Kanban no solo mejora la productividad, sino que también asegura un seguimiento adecuado y una capacidad de respuesta rápida ante cualquier cambio o desafío que surja.