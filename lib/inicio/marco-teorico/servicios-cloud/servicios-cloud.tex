\section{Servicios en la nube}
Los servicios en la nube son recursos para aplicaciones e infraestructura que existen en internet, y permiten a los clientes aprovechar recursos de computación potentes sin la necesidad de adquirir o mantener hardware o software. (Hewlett Packard Enterprise, s.f.)

Cuando se utilizan los servicios en la nube, se puede delegar la gestión de la infraestructura y centrarse en utilizarla. El proveedor que se elija podrá proporcionar una amplia gama de actividades que mantendrán un negocio en funcionamiento. Cuando se emplean estos servicios, los usuarios autorizados podrán comunicarse, colaborar y gestionar proyectos, así como analizar, procesar, compartir y almacenar datos sin la necesidad de que alguien más tenga que supervisar, mantener o realizar copias de seguridad de la actividad.
    \subsection{Beneficios de los servicios en la nube}
Los servicios en la nube ofrecen una gran variedad de beneficios para las empresas, entre los cuales se encuentran:
    \begin{itemize}
        \item Escalabilidad: Proporcionan la capacidad para escalar los recursos vertical y horizontalmente de forma instantánea en respuesta a la demanda.
        \item Flexibilidad: Se puede acceder a recursos informáticos y modificarlos con más agilidad, aumentando su capacidad para responder al uso de las unidades de negocio, los cambios en las circunstancias del mercado y las demandas de los clientes.
        \item Rentabilidad: Los mecanismos de precios con pago por consumo y la ausencia de inversiones iniciales en hardware e infraestructura mejoran la eficiencia de los costes de capital.
        \item Seguridad: Para ofrecer protección frente amenazas e infracciones, los proveedores de nube emplean fuertes características de seguridad como cifrado, límites al acceso y supervisión.
    \end{itemize}
    \subsection{Tipos de servicios en la nube}
Existen diferentes tipos de servicios que proporciona la nube, y en cada caso, los proveedores mantienen la infraestructura de la nube subyacente.
    \subsubsection{Software como servicio (SaaS)}
En este servicio, los proveedores a los suscriptores el uso de su software ejecutándose sobre la infraestructura de la nube, lo que significa que la aplicación permite una amplia distribución y acceso. El proveedor se ocupa de cuestiones como la gestión y el control de la red, los servidores, sistemas operativos, almacenamiento, virtualización, datos, middleware e incluso capacidades de aplicaciones individuales. Las aplicaciones de SaaS se suelen diseñar para resultar fáciles de usar por un público más amplio.
    \subsubsection{Plataforma como servicio (PaaS)}
Con PaaS, los usuarios tienen más control que con SaaS, porque obtienen acceso a un marco que empieza en el sistema operativo. Permite a los usuarios ubicar sus propias aplicaciones en la infraestructura de la nube con lenguajes de programación, bibliotecas, servicios y herramientas que admita el proveedor. El suscriptor dispone de control sobre las aplicaciones implementadas, los datos y, posiblemente, los ajustes de configuración del entorno de hospedaje. Sin embargo, la gestión y control de la red, los servidores, los sistemas operativos y el almacenamiento recaen en el proveedor.
    \subsubsection{Infraestructura como servicio}
Con IaaS, los suscriptores pueden diseñar un entorno completo configurando una red virtual separada de otras redes. Los usuarios ejecutan un sistema operativo y aprovisionan el procesamiento, el almacenamiento, las redes y otros recursos informáticos fundamentales para ejecutar software en la infraestructura de la nube. IaaS también proporciona a los suscriptores un control limitado de determinados de la red, y en ocasiones, tambien ofrecerán servicios como supervisión, automatización, seguridad, equilibrio de carga y resiliencia del almacenamiento.
    \section{Proveedores de servicios en la nube}
Un proveedor de servicios en la nube (CSP) es una empresa de Ti que proporciona recursos de computación bajo demanda y escalables, como potencia de computación, almacenamiento de datos o aplicaciones por internet (Google Cloud, s.f.).

El mercado de los proveedores de servicios de comunicaciones incluye una gran variedad de proveedores de servicios en la nube. Los que se consideran líderes consolidados son Google Cloud, Microsoft Azure y Amazon Web Services (AWS), sin embargo, hay muchas otras empresas pequeñas que también ofrecen servicios en la nube como IBM, Alibaba, Oracle, Red Hat, DigitalOcean y Rackspace. 

Para el desarrollo de este proyecto se utilizó Amazon Web Services, por lo que, a continuación, se hablará únicamente de esta plataforma y los servicios utilizados.
    \subsection{Amazon Web Services (AWS)}
AWS es un proveedor de servicios en la nube que ofrece una gran variedad de servicios y características que van desde tecnologías de infraestructura como cómputo, almacenamiento y bases de datos hasta tecnologías emergentes como aprendizaje automático e inteligencia artificial, lagos de datos y análisis e internet de las cosas. Esto hace que llevar las aplicaciones existentes a la nube sea más rápido, fácil y rentable y permite crear casi cualquier cosa que se pueda imaginar. (Amazon Web Services, Inc., s.f.)

La infraestructura de AWS está basada en centros de datos distribuidos en todo el mundo, lo que permite a los usuarios implementar sus aplicaciones y servicios en ubicaciones geográficas específicas según sus necesidades. Además, posee una amplia gama de herramientas de gestión y automatización para ayudar a los usuarios a administrar y controlar sus recursos en la nube de manera eficiente.
    \subsubsection{Servicios de AWS}
A continuación, se describen algunos de los servicios de AWS que se utilizaron para el desarrollo del proyecto:
    \begin{itemize}
        \item Amazon Elastic Compute Cloud (Amazon EC2): Amazon EC2 proporciona capacidad de computación escalable bajo demanda en la nube de AWS. El uso de este servicio reduce los costos de hardware para que se pueda desarollar e implementar aplicaciones con mayor rapidez. Permite lanzar cuantos servidores virtuales se necesiten, configurar la seguridad y las redes, y administrar el almacenamiento.
        \item Amazon Relational Database Service (Amazon RDS): Es un servicio de base de datos relacional fácil de administrar, optimizada para el costo total de propiedad. Es fácil de configurar, operar y escalar según la demanda. Automatiza las tareas de administración de bases de datos como el aprovisionamiento, la configuración, las copias de seguridad y la aplicación de revisiones. Permite a los clientes crear una nueva base de datos en cuestión de minutos y ofrece flexibilidad para personalizar las bases de datos a fina de satisfacer las necesidades en ocho motores.
        \item Amazon Simple Sotrage Service (Amazon S3): Es un servicio de almacenamiento de objetos que ofrece escalabilidad, disponibilidad de datos, seguridad y rendimiento. Almacena datos como objetos dentro de buckets. Un objeto es un archivo y cualquier metadato que describa ese archivo. Un bucket es un contenedor de objetos. Todos los clientes pueden usar este servicio para almacenar y proteger cualquier cantidad de datos para diversos casos de uso como lagos de datos, sitios web, aplicaciones móviles, dispositivos IoT, entre muchas otras cosas. 
    \end{itemize}