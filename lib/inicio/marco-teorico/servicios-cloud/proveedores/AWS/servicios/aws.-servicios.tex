\subsubsection{Servicios de AWS}
A continuación, se describen algunos de los servicios de AWS que se utilizaron para el desarrollo del proyecto:
    \begin{itemize}
        \item Amazon Elastic Compute Cloud (Amazon EC2): Amazon EC2 proporciona capacidad de computación escalable bajo demanda en la nube de AWS. El uso de este servicio reduce los costos de hardware para que se pueda desarollar e implementar aplicaciones con mayor rapidez. Permite lanzar cuantos servidores virtuales se necesiten, configurar la seguridad y las redes, y administrar el almacenamiento.
        \item Amazon Relational Database Service (Amazon RDS): Es un servicio de base de datos relacional fácil de administrar, optimizada para el costo total de propiedad. Es fácil de configurar, operar y escalar según la demanda. Automatiza las tareas de administración de bases de datos como el aprovisionamiento, la configuración, las copias de seguridad y la aplicación de revisiones. Permite a los clientes crear una nueva base de datos en cuestión de minutos y ofrece flexibilidad para personalizar las bases de datos a fina de satisfacer las necesidades en ocho motores.
        \item Amazon Simple Sotrage Service (Amazon S3): Es un servicio de almacenamiento de objetos que ofrece escalabilidad, disponibilidad de datos, seguridad y rendimiento. Almacena datos como objetos dentro de buckets. Un objeto es un archivo y cualquier metadato que describa ese archivo. Un bucket es un contenedor de objetos. Todos los clientes pueden usar este servicio para almacenar y proteger cualquier cantidad de datos para diversos casos de uso como lagos de datos, sitios web, aplicaciones móviles, dispositivos IoT, entre muchas otras cosas. 
        \item Elastic Load Balancing: Es un servicio que distribuye automáticamente el tráfico entrante entre varios destinos, como instancias EC2, contenedores y direcciones IP, en una o varias zonas de disponibilidad. Supervisa el estado de los destinos registrados y dirige el tráfico únicamente a los destinos en buen estado. Este servicio escala automáticamente la capacidad del balanceador de carga en respuesta a los cambios en el tráfico entrante.
        \item Amazon Route 53: Es un servicio web de sistema de nombres de dominio (DNS) escalable y de alta disponibilidad. Se puede utilizar Route 53 para realizar tres funciones principales en cualquier combinación: registro de dominios, enrutamiento DNS y comprobación de estado.
    \end{itemize}