\subsubsection{Plataforma como servicio (PaaS)}
Con PaaS, los usuarios tienen más control que con SaaS, porque obtienen acceso a un marco que empieza en el sistema operativo. Permite a los usuarios ubicar sus propias aplicaciones en la infraestructura de la nube con lenguajes de programación, bibliotecas, servicios y herramientas que admita el proveedor. El suscriptor dispone de control sobre las aplicaciones implementadas, los datos y, posiblemente, los ajustes de configuración del entorno de hospedaje. Sin embargo, la gestión y control de la red, los servidores, los sistemas operativos y el almacenamiento recaen en el proveedor.