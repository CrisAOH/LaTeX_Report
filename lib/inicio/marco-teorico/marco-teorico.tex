\chapter{Marco Teórico}
El desarrollo web es el proceso de crear y mantener sitios web y abarca una gran variedad de acciones que van desde la creación de códigos y diseños, hasta la administración de contenidos y servidores. Para el desarrollo web se utilizan diferentes lenguajes de programación que ayudan a crear el código que hace que las páginas funcionen.

Entre los lenguajes de programación más comunes para el desarrollo web se encuentran:
    \begin{itemize}
        \item HyperText Markup Language (HTML): Este lenguaje es el componente más básico de la web, y ayuda a definir el significado y la estructura del   contenido web.
        \item Cascading Style Sheet (CSS): Este lenguaje de estilos es utilizado para describir la presentación de documentos HTML o XML.
        \item Hypertext Preprocessor (PHP): Es un lenguaje de programación del lado del servidor que puede integrarse en HTML para crear aplicaciones y sitios web dinámicos.
        \item JavaScript: Es un lenguaje de programación que se ejecuta del lado del cliente de la web y es utilizado para programar cómo se comportan las páginas web cuando ocurre un evento. También puede ser ejecutado de lado del servidor, lo que permite que se puedan generar páginas web dinámicas.
    \end{itemize}
El desarrollo web se puede dividir en tres categorías: Frontend, Backend y Full Stack.

El desarrollo Frontend es el responsables de la parte del cliente, es decir, la parte que el usuario ve en pantalla cuando ingresa al sitio.

El desarrollo Backend se encarga de aspectos como servidores, bases de datos y lenguajes de programación. En esta parte se procesan las solicitudes del Frontend que contienen los datos para la base de datos u otros sitemas.

El desarrollo Full Stack es la combinación del Frontend y Backend, por lo que esta categoría es responsable tanto de la parte del cliente como de la parete del servidor.

