\section{Frameworks}
En el mundo del desarrollo (y en muchas otras áreas) existen marcos de trabajo o frameworks que ofrecen una estructura base para elaborar un proyecto con objetivos específicos; es una especie de plantilla que sirve como punto de partida. El uso de un framewrok puede ayudar a realizar un proyecto en menos tiempo gracias a la reutilización de herramientas y módulos, y en la programación, también ayuda a tener un código más limpio y consistente. (Unir FP, 2022)

Existen una gran diversidad de frameworks que son utilizados tanto para el desarrollo Frontend como el desarrollo Backend. 
    \subsection{Frameworks para Frontend}
Estos frameworks proporcionan una base para el desarrollo de la interfaz de usuario de las aplicaciones o páginas web. Hay una gran diversidad de frameworks frontend disponibles, lo que permite que se puedan crear soluciones de muy diferentes escalas y objetivos.

A continuación, se mencionan algunos de los que han sido utilizados para este proyecto:
    \subsubsection{Next.js}
Next.js es un framework JavaScript ligero y de código abierto creado sobre React que permite construir sitios y aplicaciones web rápidos y sencillos de usar. Se utilizan los componenetes de React para construir la interfaz de usuario mientras que Next.js se encarga de algunas características adicionales y optimizaciones.

Next.js también abstrae y configura automáticamente las herramientas que se requieren para React, como agrupación, compilación, entre otras cosas. Esto permite que los desarrolladores se enfoquen más en el desarrollo de la aplicación en lugar de gastar tiempo en configuraciones.
    \subsubsection{React.js}
Para algunas personas, React es un framework pero en realidad, de acuerdo al sitio oficial (https://es.react.dev/), es una librería de JavaScript. Aunque no es un framework, se describirá en esta sección por ser una herramienta de gran importancia para el proyecto.

React.js es una librería de JavaScript de código abierto desarrollada por Facebook cuyo objetivo es simplificar el proceso de construir interfaces de usuario interactivas (Herbert, 2022). Permite desarrollar las aplicaciones con la creación de componentes reutilizables; estos componentes son piezas individuales de una interfaz final.

El rol principal de React es encargarse de la capa de vista de una aplicación proveyendo la mejor y más eficiente ejecución de renderizado. Además, motiva a los desarrolladores a separar las interfaces en componentes individuales y reutilizables, de esta manera, React combina la velocidad y eficiencia de JavaScript con un método más eficiente de manipulación del DOM para renderizar las páginas web de manera más rápida.
    \subsection{Frameworks para Backend}
Estos frameworks permiten simplificar algunos aspectos del proceso de desarrollo web, haciéndolo más fácil y rápido; ayudan al desarrollador a construir la arquitectura de su sitio web.

A continuación, se mencionan algunos de los que han sido utilizados para este proyecto: 
    \subsubsection{Node.js}
Node.js no es un framework, si no un entorno de ejecución para JavaScript, A pesar de esto, se describirá en esta sección debido a su importancia para el desarrollo del proyecto.

Node.js es un entorno de ejecución de JavaScript de código abierto y multiplataforma que permite a los desarrolladores crear servidores, aplicaciones web, herramientas para la línea de comandos y scripts.

Está basado en el motor de código abierto V8 de Google, el cual se actualiza constantemente y ofrece una gran rapidez, es por ello que se recomienda para el desarrollo de aplicaciones en tiempo real, las cuales se caracterizan por proporcionar información de manera instantánea.
    \subsubsection{Express.js}
Express.js es un framework de backend para Node.js que proporciona características y herramientas robustas para desarrollar aplicaciones de backend escalables (OpenJS Foundation, s.f).

Este framework proporciona un conjunto de herramientas para aplicaciones web, peticiones y respuestas HTTP, enrutamiento y middleware para construir y desplegar aplicaciones a gran escala. Es utilizado para una gran variedad de cosas en el ecosistema JavaScript/Node.js; se pueden desarrollar aplicaciones, endpoints de APIs, sistemas de enrutamiento y frameworks (Kinsta, 2022)

Express.js es un framework no dogmático, es decir, un framework que no impone demasiadas restricciones sobre la mejor manera de unir componentes para alcanzar un objetivo. Es por esto que permite insertar casi cualquier middleware compatible dentro de la cadena del manejo de la petición, en cualquier orden; además, permite definir una estructura de directorios propia, por lo que se pueden crear tantas carpetas y archivos como se desee (MDN Web Docs, s.f.).