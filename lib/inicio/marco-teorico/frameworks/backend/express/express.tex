\subsubsection{Express.js}
Express.js es un framework de backend para Node.js que proporciona características y herramientas robustas para desarrollar aplicaciones de backend escalables (OpenJS Foundation, s.f).

Este framework proporciona un conjunto de herramientas para aplicaciones web, peticiones y respuestas HTTP, enrutamiento y middleware para construir y desplegar aplicaciones a gran escala. Es utilizado para una gran variedad de cosas en el ecosistema JavaScript/Node.js; se pueden desarrollar aplicaciones, endpoints de APIs, sistemas de enrutamiento y frameworks (Kinsta, 2022)

Express.js es un framework no dogmático, es decir, un framework que no impone demasiadas restricciones sobre la mejor manera de unir componentes para alcanzar un objetivo. Es por esto que permite insertar casi cualquier middleware compatible dentro de la cadena del manejo de la petición, en cualquier orden; además, permite definir una estructura de directorios propia, por lo que se pueden crear tantas carpetas y archivos como se desee (MDN Web Docs, s.f.).