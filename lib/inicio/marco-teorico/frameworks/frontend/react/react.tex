\subsubsection{React.js}
Para algunas personas, React es un framework pero en realidad, de acuerdo al sitio oficial, es una librería de JavaScript. Aunque no es un framework, se describirá en esta sección por ser una herramienta de gran importancia para el proyecto.

React.js es una librería de JavaScript de código abierto desarrollada por Facebook cuyo objetivo es simplificar el proceso de construir interfaces de usuario interactivas. Permite desarrollar las aplicaciones con la creación de componentes reutilizables; estos componentes son piezas individuales de una interfaz final.

El rol principal de React es encargarse de la capa de vista de una aplicación proveyendo la mejor y más eficiente ejecución de renderizado. Además, motiva a los desarrolladores a separar las interfaces en componentes individuales y reutilizables, de esta manera, React combina la velocidad y eficiencia de JavaScript con un método más eficiente de manipulación del DOM para renderizar las páginas web de manera más rápida.