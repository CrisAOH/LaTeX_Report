%Referencias renombrado
\renewcommand{\bibname}{Bibliografía}

%Configurando parskip
\setlength{\parindent}{0cm}
\setlength{\parskip}{5mm}

%Configurando el espaciado entre "itemize/enumerate items"
\setlist[itemize]{noitemsep, topsep=0mm}
\setlist[enumerate]{noitemsep, topsep=0mm}

%Espacios entre autores en "Bibliografía"
\let\OLDthebibliography\thebibliography{}
\renewcommand\thebibliography[1]{
    \OLDthebibliography{#1}
    \setlength{\parskip}{3mm}
    \setlength{\itemsep}{3mm plus 0.3ex}
}

%Numeración en los reportes
\setcounter{secnumdepth}{3}

%Numeración en el índice
\setcounter{tocdepth}{3}

%Elimación del salto de página del comando de capítulos
\makeatletter
\renewcommand{\chapter}{
    \vspace{60pt}
    \if@openright\cleardoublepage\else\par\fi % Esto causa el salto automático; lo eliminamos.
    \global\@topnum\z@
    \@afterindentfalse
    \secdef\@chapter\@schapter
}
\makeatother

%Numeración y texto de las figuras
\renewcommand{\figurename}{Figura}
\renewcommand{\thefigure}{\arabic{figure}}
\captionsetup{font=footnotesize}


%Cambiar el interlineado del texto
\setstretch{1.5}

%Cambiar el título del índice
\renewcommand{\contentsname}{Índice}

%Añadir "título" arriba de las páginas
\addtocontents{toc}{~\hfill\textbf{Pág.}\par}

%Cambiar estilo de los títulos de los capítulos, secciones, subsecciones, párrafos, etc.
\titleformat{\chapter}[hang]{\normalfont\LARGE\bfseries\setstretch{1.0}}{\thechapter.}{10pt}{\LARGE\bfseries}
\titleformat*{\section}{\Large\bfseries}
\titleformat*{\subsection}{\large\bfseries}
\titleformat*{\subsubsection}{\normalsize\bfseries}
% \titleformat{\paragraph}{\normalsize\bfseries}

%Cambiar el espacio que hay entre el título del capítulo y eñ margen
% \titlespacing*{\chapter}{0pt}{-40pt}{0pt}
\titlespacing*{\chapter}{0pt}{-40pt}{0pt}
\titlespacing*{\subsection}{0pt}{15pt}{0pt}
\titlespacing*{\subsubsection}{0pt}{15pt}{0pt}